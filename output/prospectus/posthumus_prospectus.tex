\documentclass{article}
\usepackage[portrait, margin=1in]{geometry}
\usepackage{graphicx} 
\graphicspath{{/Users/danpost/housing_project/output/prospectus/graphics}}
\usepackage{amsmath}
\usepackage{hyperref}
\usepackage{booktabs}
\usepackage{array}
\usepackage[backend=bibtex]{biblatex}
\addbibresource{prospectus_bib.bib}

\begin{document}

\title{Housing Policy and the Construction of Political Coalitions} 
\author{Daniel Posthumus}

\maketitle

\section{Introduction}

In the wake of the 2024 presidential election, the liberal movement in the United States has become fascinated with the ``abundance agenda". The premise of the agenda is simple: we don't have enough \textit{stuff}. Then, according to Economics 101, if supply is constrained then things will be more expensive. The forefront of the agenda is housing; the United States is fortunate to have high immigration and resulting population growth despite our wealth, but housing constraints have reduced growth in housing supply in the most productive places the country. This has made housing unaffordable for a whole generation of young people. 

This challenge has a political component and an economic component, both of which have been extensively studied. There's a mismatch between these two forces: the economics have driven demand for housing, while the political has choked supply, keeping it  far below where market forces would drive it. These forces have been extensively studied separately; however, bridging the gap between the two is an open challenge for scholars. 

One reason why we don't understand a full model of housing is that political outcomes in housing policy are endogenous; housing policy is determined via political processes which involve residents of political constituencies, the composition of which are determined by who chooses and affords to live there. 

There is a third actor in this process: banks. Owning a house requires a mortgage, which requires lending from a bank. The prospect of owning a house is critical to the idea of the American Dream, one reason why housing policy and economics have such political salience in local elections. Mass ownership of housing requires mass capital available \textit{from banks}; yet, we don't know what role banks play in formulating housing policy and designing housing markets. 

\section{Literature Review}



\section{Data}



\section{Reduced Form Evidence}



\section{Conclusion}



\nocite{*}

\printbibliography

\end{document}