\documentclass{article}
\usepackage{graphicx} % Required for inserting images

\title{Estimating the Political Feedback Loop of Local Housing Politics}
\author{Daniel Posthumus}

\begin{document}

\maketitle

\tableofcontents

\section{Introduction}



\section{Literature Review}


\section{Institutional Background}


\section{Empirical Methods}

\subsection{Existing Data on Housing Regulations}
This project requires leveraging both existing data and the creation of a new dataset. First, a granular time-series of the regulatory stringency of housing policy on the community level doesn't exist. Any researcher seeking to use housing policy stringency as a model input must choose between the survey-based Wharton Land Use Regulatory Index and Bartik, Milo, and Gupta (2024)'s LLLM-created dataset (hereafter referred to as the `LLM Index'). The Wharton Index has observations from two time periods: 2005 and 2018, whereas the LLM Index only has one period of observation -- 2024. Also unfortunately, the two don't match in where they place communities in the distribution of regulatory stringency. And because their time periods don't match perfectly, it is impossible to \textbf{definitively rule out} that some localities significantly changed their place in the distribution of housing policy stringency between 2018 and 2024. 

This means estimating a model of feedback by using these indices is not possible; we need a time-series related to housing regulation on a more granular level to match high-frequency political changes. Ideally, we also want a panel dataset, composed of interconnected housing markets. To illustrate why this is necessary is straightforward. Imagine individual $i$ prefers a low density community. They live in San Francisco, denoted as $SF$. They vote for a NIMBY candidate in period $t$. Then, their candidate loses and the planning commission uses its authority to approve a spate of new projects in $i$'s neighborhood. Individual $i$ then moves to Daly City, in San Mateo County. Intuitively, individual $i$ hasn't changed housing markets; there is a clear and direct relationship between Daly City real estate and San Francisco real estate. If individual $i$ would have moved to Stockton, California, however, we could effectively argue for strong geographical segmentation between the San Francisco and Stockton housing markets. 

To model this example sketch, we need data on community housing policy at a high-frequency level, and for the polities in not just in the San Francisco city boundaries, but more broadly contained in the San Francisco metropolitan area, say the Bay County. The most straightforward way to do this is to dive into the meeting minutes of the planning bodies for various communities in the Bay Area. The most prominent and simplest example here would be San Francisco's planning commission, which controls the levers of approval power for projects and permitting within the city boundaries of San Francisco. 

\subsection{Creation of Planning Commission Dataset}



\end{document}